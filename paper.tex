%-----------------------------------------------------------------------------
%
%               Template for sigplanconf LaTeX Class
%
% Name:         sigplanconf-template.tex
%
% Purpose:      A template for sigplanconf.cls, which is a LaTeX 2e class
%               file for SIGPLAN conference proceedings.
%
% Guide:        Refer to "Author's Guide to the ACM SIGPLAN Class,"
%               sigplanconf-guide.pdf
%
% Author:       Paul C. Anagnostopoulos
%               Windfall Software
%               978 371-2316
%               paul@windfall.com
%
% Created:      15 February 2005
%
%-----------------------------------------------------------------------------


\documentclass[preprint]{sigplanconf}

% The following \documentclass options may be useful:
%
% 10pt          To set in 10-point type instead of 9-point.
% 11pt          To set in 11-point type instead of 9-point.
% authoryear    To obtain author/year citation style instead of numeric.

\usepackage{amsmath}

\begin{document}

\conferenceinfo{Haskell Symposium 2012}{13th September, Copenhagen.}
\copyrightyear{2012}
\copyrightdata{[to be supplied]}

\titlebanner{A draft paper to be submitted to Haskell Symposium 2012}
\preprintfooter{TBD: short description of paper}   % 'preprint' option specified.

\title{Behavior Driven Development in Haskell}
%\subtitle{Subtitle Text, if any}

\authorinfo{Simon HENGEL}
           {TBD: Affiliation}
           {TBD: Email}
\authorinfo{Trystan SPANGLER}
           {TBD: Affiliation}
           {TBD: Email}
\authorinfo{Kazuhiko YAMAMOTO}
           {IIJ Innovation Institute Inc.}
           {kazu@iij.ad.jp}

\maketitle

\begin{abstract}
TBD: This is the text of the abstract.
\end{abstract}

\category{TBD:CR-number}{subcategory}{third-level}

\terms
TBD

\keywords
software design, documentation, automatic test

\section{Introduction}

The contribution of this paper is as follows:

\begin{itemize}
\item Introduced doctest from Python community to Haskell community
\item Introduced *spec from Ruby community to Haskell community
\item Integrated doctest and QuickCheck
\item Show best current practice of design/documentation/tests
\end{itemize}

\section{Terminology}

\begin{description}
\item[unit test] TBD
\item[example] TBD
\item[property] TBD
\item[behavior] TBD
\item[documentation] Text written in Haskell comments and parsed by Haddock
\end{description}

\section{Observation}

Should talk about equation property.

\section{Solution}

The authors are not fundamentalist.
If functions to be defined are simple enough,
we might implement them first withtout signatures.
And their signatures can be automatically
inserted by IDE (like ghc-mod) later
if Haskell compilers can infer correctly.
Also, we can cut and paste typical examples and their results
from Haskell interpreters to their document.

\section{Implementation}

\subsection{doctest}

\subsection{doctest and QuickCheck}

\subsection{hspec}

\section{Conclusion}

%% \appendix
%% \section{Appendix Title}

%% This is the text of the appendix, if you need one.

\acks

TBD: Acknowledgments, if needed.

% We recommend abbrvnat bibliography style.

\bibliographystyle{abbrvnat}

% The bibliography should be embedded for final submission.

\begin{thebibliography}{}
\softraggedright

\bibitem[Smith et~al.(2009)Smith, Jones]{smith02}
P. Q. Smith, and X. Y. Jones. ...reference text...

\end{thebibliography}

\end{document}
